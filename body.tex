\summary

Recommendation systems are the ground in which successful social and e-commerce website are based. The concepts of `Like` and `Suggested` heavily used in services like Facebook, Amazon, Google, Netflix, Hulu, Bravofly are based on the concept that users of a system may like or dislike an item. The collection of this preferences or ratings is the basic data for any recommendation algorithms which goal is to find the next item that the user may like. This feature is used in different ways depending from the business model that the service uses: for example on an e-commerce website such as Amazon it is used to let the user think about adding the recommended item in cart to maximise profits, on Netflix the recommendation are used to keep people using the system and thus increase user fidelity.

Recommendation systems are used also to restrict the number of items to display to the user to overcome to information overloading problems. In big e-commerce website, having a big catalogue of thousands of items may disorient the user. Using a recommendation the e-commerce site may still offer a way to navigate through a thousand item catalog, maybe with the support of a search engine, and show the items that the user may look for based on previous purchases of the same user.

Recommendation systems are also used for marketing purposes. Companies like Google use the information of what the user searched to promote some advertisements in respect of others. In this scenario what the user `liked` is what the user searched for and since the user took an active part in looking for something the advertisement will be much more effective and likely to convince the user in selecting the advertised item.

Recommendation systems differentiate between each other from the used algorithm. The recommendation algorithm is the basic software that makes a recommendation system perform better than another. It takes three sets of elements: users, items and ratings. It then outputs a list of items for each user. The output is also called recommendation. The process of running an algorithm to generate a recommendation is a computational intensive process and thus complex algorithms elaborate a model and then they use this model to generate the actual recommendation. Creating a model is a very intense process that requires a lot of computation power and RAM for several days. After a model is created it can be used quickly on demand to get the list of suggested items for an user. The goal of this thesis is to build a web application that is flexible enough to allow recommendation system experts to try and experiment with new algorithms easily.

The work of this thesis is based on a previous recommendation system developed by me and Andrèia Coronado Cha which was previously based on the ContentWise \cite{ContentWise} software by Moviri. 

ContentWise is and innovative content recommendation engine for IPTV providers. Contentwise has been adopted by a major European triple-player Telco and shortlisted at the IPTV World Series Awards 2008 at the IPTV World Forum.
In 2007 Moviri won the first edition of the Start Up of the Year Award, promoted by PNICube, (the Italian Association of Universities Incubators) for its innovative business and potential growth. More recently, Moviri has partnered with major vendors and service providers as HP, BMC, IBM, Oracle and VMware. In 2008 Moviri has been named HP BTO Partner of the year.     

Even if ContentWise has been successfully adopted by a wide range of companies it is not suitable for researches in new recommendation algorithms. To overcome this limitation the first system, called Milo, was developed during the digital and internet television Politecnico di Milano course. Milo was a first attempt in experimenting a new graphical interface for the user and was relaying in too many different technologies that made the overall system really fragile.
Based on the knowledge developed during the coding of Milo, I've developed another recommendation system called Movish which differentiate from the previous one by the following factors:
\begin{itemize}
\item Technology. Movish is built on top of web2py\cite{web2py}, a free open source full-stack framework for rapid development of fast, scalable, secure and portable database-driven web-based applications under the GPLv3\cite{gplv3} license. All the computational part is managed by Matlab which talks with the web2py powered web applicaton thanks to the pymatlab\cite{pymatlab} library. Pymatlab is basically a layer that translates python data structures in Matlab ones. 
\item Integration. The system is a single application and not two separate projects like in Milo. This allows a better control of the system and a single entry point for interventions. The application is still modular but there is not a project level separation between the web application and the recommendation engine.
\item Flexibility. The system is able to recognize if the administrator adds new algorithms to the source tree and updates the admin panel accordingly. The administrator is thus able of generating a model for a given algorithm and create new survey without modifying the system code. This allows researches to be able to use the system for their tests without caring about the implementation of the system that is going to use the algorithm they implemented. This makes Movish an effective recommendation system for researching new algorithms or improving existing ones. 
\item Independence. The system has an embedded crawler that crawls the information about movies from imdb \cite{imdb}. Every week the system automatically fetches all the movies that have been released or that are scheduled to be released in an interval of 5 years from the given week. When displaying a movie if some information about the movie is missing the system tries to retrieve the missing information from imdb. The administrator can order to update a movie, the whole dataset or retrieve the most popular movies or the coming soon movies from imdb anytime thanks to the adminstration interface.
\item Scalability. The system has been developed with the cloud in mind. Everything is self contained in a virtualized KVM image which can be easily deployed on popular cloud infrastructure like Amazon EC2 and Rackspace. It uses the web2py scheduler heavily to allow asynchronous tasks. Every time consuming operation is encapsulated in an asynchronous task which is executed by a list of workers without affecting web users with slow navigation through the movie catalog. 
\end{itemize}

\bibliographystyle{formb}
\bibliography{main}



