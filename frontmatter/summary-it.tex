% !TEX root = ../thesis.tex
\chapter*{Sommario}

I sistemi di raccomandazione sono, al giorno d'oggi, componente fondamentale di tutti i nuovi servizi di successo come Facebook, Amazon, Google, Netflix e Hulu. Tutti questi servizi collezionano in maniera pi\'u o meno pervasiva tutte le nostre preferenze, ovvero tutti i \textit{like} che aggiungiamo ai contenuti che ci vengono proposti. Queste informazioni sono gestite ed utilizzate in maniera differente che dipende dal servizio: per esempio un servizio di ecommerce come Amazon utilizza i dati su che oggetti ci interessano per selezionare altri oggetti di nostro interesse e proporli durante l'acquisto o la visualizzazione del sito in modo di massimizzare le vendite e quindi i profitti.

I sistemi di raccomandazione sono anche utilizzati per diminuire il numero di oggetti visualizzati in caso di sistemi con un vasto catalogo. Infatti, se Amazon dovesse mostrare tutto il suo catalogo in una pagina l'utente sarebbe molto disorientato. Un sistema di raccomandazione è in grado di scegliere gli oggetti pi\'u pertinenti per un utente e quindi visualizzare quelli in primo piano in modo da non disorientare l'utente e suscitare interesse nell'utente.

I sistemi di raccomandazione sono anche utilizzati nel campo del marketing. Aziende come Google utilizzano le preferenze degli utenti per visualizzare o meno una determinata pubblicità in modo da massimizzare, anche in questo caso, le vendite e quindi i profitti.

I vari sistemi di raccomandazione si differenziano principalmente per gli algoritmi utilizzati. Gli algoritmi di raccomandazione sono il cuore di una raccomandazione e decidono quali elementi selezionare tra quelli in catalogo. Ogni algoritmo ha come input gli utenti, gli elementi e le preferenze o ratings degli utenti sugli elementi. Ogni algoritmo ha come output una lista di elementi consigliati per ogni utente. Il processo di generazione di una raccomandazione è molto oneroso dal punto di vista computazionale per questo gli algoritmi di raccomandazione generano prima un modello e poi riuntilizzano il modello generato con il profilo dell'utente, ovvero la lista di tutti i rating dell'utente, per generare una raccomandazione. La creazione del modello è il task pi\'u oneroso per una raccomandazione e richiede molta CPU e RAM.

Un buon algoritmo è talmente importante per una raccomandazione che il gigante del noleggio di film online, Netflix, ha nel 2006 aperto una copetizione per trovare un algoritmo che riuscisse a suggerire vari film in maniera migliore dell'algoritmo usato all'epoca. La competizione è nota come il \textit{NetFlix prize} \cite{netflixprize} e aveva un monte premi di un milione di dollari. La competizione è inziata il 2 Ottobre 2006 ed è stata conclusa il 26 Luglio 2009. L'algoritmo vincente migliorava del 10\% le raccomandazioni dell'algoritmo ufficialmente utilizzato da NetFlix all'epoca.

Il lavoro di questa tesi si basa in un lavoro precedente mio e di Andrèia Coronado Cha di nome Milo che a sua volta si basava su uno studio del software ContentWise \cite{ContentWise} di Moviri.

ContentWise è un innovativo sistema di raccomandazione per provider di servizi televisivi. Il sistema è adottato dai maggiori player di telecomunicazioni in Europa e ha ricevuto diversi riconoscimenti come l'IPTV World Series Awards 2008. Nel 2007 Moviri ha vinto la prima edizione del concorso Start Up of the year award, promosso da PNICube (associazione italiana di incubatori universitari) per il business innovativo e crescita potenziale. Recentemente Moviri ha stretto partnerships con HP, BMC, IBM, Oracle e VMware. Nel 2008 Moviri è stata nomitata HP BTO Partner of the year.

Anche se ContentWise si comporta molto bene per l'ambito per il quale è stato pensato il sistema risulta poco flessibile nel campo della ricerca. La necessità di un sistema pi\`u flessibile per svolgere ricerche sui sistemi di raccomandazione \`e stata necessaria. Il primo progetto che ha provato a dar vita ad un sistema simile a ContentWise ma pensato per far ricerca \`e stato sviluppato durante il corso di \textbf{digital and internet television} del Politecnico di Milano. Cos\`i \`e nato il progetto Milo che per\`o \`e risultato fragile durante i suoi primi utilizzi a causa delle diverse tecnologie utilizzate nello stesso applicativo e a causa della mancanza di varie ottimizzazioni atte a ridurre il consumo di CPU e RAM degli algoritmi di raccomandazione.

Con l'esperienza maturata durante il progetto Milo, questo elaborato presenta un nuovo sistema di raccomandazione per film chiamato Movish che si differenzia dal precedente per le seguenti caratteristiche
\begin{itemize}
\item \textbf{Tecnologia}. Movish si basa principalmente sul framework web2py \cite{web2py}, un framework open source per lo sviluppo rapido di applicazioni web veloci, scalabili e sicure. Tutta la parte computazionale è basata su Matlab \cite{matlab} che comunica con il framework web2py e quindi con l'applicazione tramite la libreria pymatlab \cite{pymatlab}. Pymatlab \`e essenzialmente un layer che traduce le strutture dati python in strutture dati matlab e viceversa.
\item \textbf{Integrazione}. Il sistema \`e una singola applicazione e non due progetti separati che collaborano tra di loro come in Milo. Questo permette un migliore controllo del sistema ed un singolo punto di intervento per le modifiche. Tuttavia l'applicazione rimane comunque modulare ma non esiste una separazione netta come in Milo per quanto riguarda l'applicazione web che si occupa dell'interfaccia utente con il sistema di raccomandazione che si occupa di generare le raccomandazioni.
\item \textbf{Flessibilit\`a}. Il sistema \`e in grado di riconoscere se l'amministratore aggiunge nuovi algoritmi al sistema e automaticamente importa il nuovo algoritmo nel sistema. L'amministratore di sistema \`e quindi in grado di aggiungere un nuovo algoritmo senza dover modificare il codice dell'applicazione e abilita la possibilit\`a di fare facilmente ricerca su nuovi algoritmi. 
\item \textbf{Indipendenza}. Il sistema ha funzionalit\`a di auto importazione di nuovi film da imdb.com e youtube.com per quanto riguarda i trailers degli stessi. Ogni settimana il sistema importa automaticamente tutti i film che sono stati rilasciati o che verranno rilascitati in un intervallo di 5 anni prima o 5 anni dopo della data in cui avviene l'import. Quando il sistema deve visualizzare un film in cui alcune infomazioni sono incomplete automaticamente prende le informazioni da imdb.com. L'amministratore pu\`o ordinare il refresh di tutto il dataset o importare i coming soon attuali o i pi\`u popolari selezionando la relativa funzione nel pannello di amministrazione.
\item \textbf{Scalability}. Il sistema \`e stato sviluppato per evolversi nel cloud. L'intero sistema \`e contenuto in una macchina virtuale di tipo KVM che pu\`o facilmente essere spostata da infrastruttura ad infrastruttura come Amazon EC2 o Rackspace. Inoltre l'applicazione utilizza in maniera estensiva anche lo scheduler di web2py per permettere l'esecuzione di task asincroni. Ogni operazione che richiede un lavoro computazionale non indifferente \`e incapsulata in un task che viene eseguito in maniera asincrona rispetto alle richieste web.
\end{itemize}

Grazie a questi fattori e alla conoscenza sviluppata durante un intero anno nel campo della raccomandazione, Movish risulta essere un sistema flessibile ed efficiente per sviluppare e testare nuovi algoritmi di raccomandazione. Inoltre, affinch\`e sia possibile testare nuovi algoritmi in maniera efficace un sottosistema di questionari \`e stato integrato in modo da avere feedback puntuali sulle performances o altri aspetti degli algoritmi di raccomandazione testati.

Il sottosistema di questionari \`e stato pensato per essere il pi\`u modulare possibile in modo da permettere l'inserimento di nuove tipologie di questionario in maniera semplice. La creazione di nuovi questionari la cui tipologia \`e gi\`a definita \`e invece facilmente gestibile interamente graficamente dal pannello di amministrazione.

Il capitolo \ref{chapter:<recommendation_system_state_of_the_art>} espone lo stato dell'arte di un sistema di raccomandazione; il capitolo \ref{chapter:movish_system_background} analizza ContentWise, Milo e Movish ed evidenzia le differenze tra i diversi sistemi; nel capitolo \ref{chapter:movish_system} Movish \`e analizzato a livello architetturale e nel dettaglio; nel capitolo \ref{chapter:<survey_management>} il sottosistema dei questionari \`e analizzato ed infine nel capitolo \ref{chapter:<research_number_of_items>} la prima ricerca utilizzando Movish \`e esposta. 
 
La ricerca, sempre pubblicata in questo elaborato, che vuole correlare la capacità di una raccomandazione di essere percepita come utile da un utente in relazione al numero di film della raccomandazione stessa e rileva risultati interessanti.

\acresetall